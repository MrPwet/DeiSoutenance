\chapter{Le financement de l'open source}

Beacoup de projets libres sont gratuits, mais cela ne signifie pas forcément
que rien n'a été investi : il existe de nombreux modèles de financement des
projets open source dont certains initiés par des sociétés comme RedHat ou Apple
qui sont en croissance constante. Le logiciel libre peut donc être rentable à
condition de savoir exploiter le modèle de financement adapté à l'ampleur du
projet.

\section{Quelques modèles alternatifs}

    \subsection{Les dons}

Le financement de projets par dons est le premier type de financement du 
logiciel libre a être apparu, et il est encore très répandu.
Ce modèle très simple permet à un utilisateur convaincu de l'utilité du 
produit de faire un don ponctuel, généralement par le biais de paypal.
Si ce modèle peut servir d'appoint aux développeurs, il n'est viable que dans 
très peu de cas.

    \subsection{Modèle openBSD}

Le modèle utilisé par le projet openBSD, un système d'exploitation de la 
famille des BSD consiste à utiliser deux moyens de distribution : d'une part 
une distribution gratuite en téléchargement sur internet, et d'autre part une 
distribution sur support physique, accompagnée de goodies tels que des posters,
des autocollants et un copie d'une chanson décrivant l'avancée du projet.
Malgré cet apparence de vente, à l'analyse on se rend compte qu'il s'agit
d'une forme avancée du modèle basé sur les dons : en effet, les acheteur du
modèle physique, pour la plus grande majorité possèdent et utilisent déjà le 
produit. La raison qui les pousse à acheter le produit qu'ils possèdent déjà 
ne vient pas d'un besoin mais d'un désir d'aider financièrement le projet.

    \subsection{Donations par micropaiement}

Le modèle du pay per like est une version plus organisée du modèle par dons.
Les dons se font non plus directement au projet mais passent par un oganisme
chargé de répartir équitablement un budget mensuel, défini par le donneur,
entre tous les projets qu'il aura explicitement marqués dans le mois.
\url{flattr.com}, par exemple, permet distribuer le budget entre les projets
marqués sur \url{github.com}. Flattr, bien que compatible pour le financement
de projets informatiques est surtout prévu pour récompenser les créateurs de
contenu artistique.

    \subsection{Le Crowdfunding}

\paragraph{} En français, on l'appelle "financement collaboratif". Il s'agit
d'un modèle de financement basé sur la collecte de don. Le leader sur le marché
est le site Kickstarter. Il propose à un utilisateur ou un groupe d'utilisateur
de mettre en ligne l'idée de son projet pour qu'il apparaisse sur le site. Les
visiteurs peuvent alors consulter cette idée et choisir ou non d'investir de
l'argent pour la voir se développer. Les personnes ayant contribué au
développement sont très souvent récompensées lors de la fin du projet.
Elles peuvent
par exemple rencontrer les créateurs du projet qu'elles ont soutenu, obtenir des
goodies\footnote{\textbf{goodies :} produits dérivés faisant référence à un
film, un jeu-vidéo\ldots} ou même pouvoir voter pour l'implémentation de
fonctionnalités.  Ce mode de financement développe l'entraide et donne la
capacité financière afin de mener à bien un projet prometteur.

    \subsection{Licence Double}

La licence double est un modèle adopté notamment par MySQL consiste à permettre
l'utilisation de son produit différemment suivant l'usage. Dans le cas de
MySQL, l'utilisation personnelle ou l'inclusion dans un projet sous licence GPL
est gratuite tandis qu'une utilisation professionnelle est payante.

    \subsection{Modèle Redhat}

Redhat distribue son système d'exploitation gratuitement mais propose des 
services associés payants, notamment l'installation, la maintenance et le 
support du logiciel dans le cas d'usage de l'entreprise contractante, 
garantissant la stabilité de son produit.
Ce modèle a été adopté par l'entreprise en 2003, qui depuis cette date est en 
constante croissance.
C'est un modèle similaire qui régit l'entreprise Canonical, éditrice du 
système d'exploitation Ubuntu qui est aujourd'hui un des systèmes dérivés de 
linux les plus utilisés.

    \subsection{Modèle Apple}

Le système d'exploitation OSX d'apple utilise des logiciels et technologies 
open source telles que des utilitaires et modules du projet freeBSD\@.
Ceux ci sont ensuite vendus aux consommateurs sous leur licence propriétaire. 
Même si cette pratique est totalement légale elle ne contribue pas à apporter 
un soutien financer aux développeurs originels des logiciels concernés.

    \subsection{Modèle Bounty}

Il existe un grand nombre de sites, tel que \url{bountysource.com}, le 
plus connu d'entre eux, permettant d'offrir des récompenses à quiconque 
travaillera sur un projet pour le maintenir, le réparer ou y ajouter des 
fonctionnalités
Les utilisateurs peuvent se regrouper sur une même demande pour augmenter la 
prime attribuée à la solution dont ils ont besoin.
Ainsi, plus une fonction intéresse d'utilisateurs, plus la somme proposée pour 
son implémentation est élevée et donc susceptible d'intéresser un développeur.

    \subsection{Modèle Pay What You Want}

    \paragraph{} Le modèle Pay What You Want consiste à proposer un produit à
    un utilisateur pour la somme qu'il estime nécessaire de payer. Il peut y
    avoir un seuil minimum à verser ou bien non. Dans le deuxième cas,
    l'utilisateur peut donc acquérir le produit gratuitement de façon légale,
    mais il peut aussi verser un montant qu'il juge adapté au produit de façon
    à récompenser le ou les concepteurs pour leur travail.

    \paragraph{} Ce type de modèle s'est démocratisé en 2007 grâce au groupe de
    musique Radiohead qui a proposé sur son site officiel son nouvel album
    intitulé "in rainbow" afin de lutter contre le piratage. On remarque
    l'utilisation de ce modèle de financement et de ses nombreux dérivés (Pay
    what you wish, Pay what you can etc.) surtout dans le secteur des
    logiciels, de la musique, de la restauration et de l'hôtellerie. On peut en
    voir un parfait exemple à l'adresse suivant
    \url{https://www.humblebundle.com/}.


\section{Gittip confronté aux autres modèles}

\subsection{La stabilité des dons}

\paragraph{}
Une part importante de la comparaison entre Gittip et d'autres modèles de
financement concerne la stabilité des revenus. On peut en effet constater qu'un
utilisateur de Gittip qui souhaite apporter sa contribution à un développeur ou
bien à une équipe de développeurs le fait de façon régulière car en
choississant
Gittip, il s'engage à verser une même somme d'argent toutes les semaines.
L'utilisateur peut bien entendu arrêter de verser cette somme dès qu'il le
désire. Dans cette façon de faire, Gittip se démarque des autres modes de
financement qui repose davantage sur des versements ponctuels.

\paragraph{}
Si on prend l'exemple de Kickstarter, les utilisateurs vont choisir de verser
une certaine somme d'argent de façon ponctuelle et la plupart du temps unique.
Le financement d'un projet lancé sur Kickstarter est donc assez imprévisible.
On peut en effet récolter de très grosses sommes en l'espace de très peu de
temps
(gros dons) ou bien alors une absence totale de don pendant une certaine
période, contrairement à Gittip qui assure un revenu régulier et connu à
l'avance de façon hebdomadaire.

\paragraph{}
La complémentarité des deux modèles précédemment cités peut être très
intéressante. On
imagine en effet parfaitement le financement d'un projet via la plateforme de
Kickstarter (qui requiert une somme souvent importante d'argent) puis le suivi
de ce projet en attribuant un revenu hebdomadaire pour subvenir au besoin
humain du développeur et lui permettre de consacrer son temps à la réalisation
de son projet grâce à Gittip.

\paragraph{}
D'autres modèles en revanche s'éloignent plus de cette problèmatique de
financement. On peut notamment citer le modèle Pay What You Want qui est
beaucoup plus adapté pour vendre le produit fini que pour aider à la conception
de ce dernier. De plus l'argent gagné lors de la vente d'un produit ne sert pas
directement à un développeur de ce produit mais il va à la société qui l'a
conçu. De ce fait on s'éloigne complètement de Gittip car on s'éloigne de la
notion de salaire. L'argent récolté par la société servira à maintenir le
projet, en créer de nouveaux ou bien à payer les salariés.

\paragraph{}
Le modèle actuellement existant qui se rapproche le plus de Gittip est le site
Flattr. On y retrouve effectivement la même notions de micro paiements à
l'intention d'une personne ou d'une équipe. La différence majeur étant que les
paiements sur Flattr ne sont pas réguliers. On peut choisir un montant à donner
pour un mois et le répartir entre plusieurs personnes. De ce fait le montant
reçu par une personne inscrite sur Flattr peut varier de façon importante
suivant les mois.
\subsection{Couts}

Si Gittip, grâce à son open company \emph{Gittip LLC}, ne souhaite pas faire de
profit, d'autres services basent leur modèle financier sur une marge prélevée à
chaque réussite d'un projet. Voici un tableau récapitulatif des marges fixées
par plusieurs services connus :

\begin{figure}[h!]
    \center{
    \begin{tabular}{|c|c|c|c|c|c|}
        \hline
        \textbf{Paypal} & \textbf{Balanced Payments} & \textbf{Gittip} &
        \textbf{Indiegogo} & \textbf{Kickstarter} & \textbf{Flattr}\\
        \hline
        $2,2\%{} + 30$\textcent & $2,9\% + 30$\textcent &  $2,9\% +
        30$\textcent & $4\%{}$ & $5\%{}$ & 10\%{}\\
        \hline
    \end{tabular}}
    \caption{Comparatif des prélèvements pas plateforme}
\end{figure}

Paypal, le service de micropaiements en ligne,  est beaucoup utilisé pour des
dons spontanés ou les paiements sur internet, il est donc intéressant d'en
savoir le cout. Balanced Payments, le système de micropaiements en ligne utilisé
par Gittip, fonctionne sur un modèle similaire à celui de Paypal. Gittip évite
les transactions inutiles, et fait la différence entre ce qu'un utilisateur
doit et ce qu'il reçoit avant de faire le virement. Ainsi si un utilisateur
reçoit autant qu'il donne, il n'aura à payer aucune charge. Kickstarter et
Indiegogo, les deux sites les plus connus pour du crowdfunding, s'attribuent
une marge relativement importante, surtout qu'il s'agit de grosses sommes
échangées sur ces services : certains projets, comme le projet Pebble en 2010
sur Kickstarter, peuvent monter jusqu'à 10 millions de dollars.

\paragraph{}
Dans le cas du crowdfunding, les marges ne sont prélevées que dans le cas où le
projet réussi à être financé. Dans le cas d'Indiegogo, si la somme fixée au
départ n'est pas atteinte à la fin du temps imparti, l'équipe qui travaille sur
le projet a alors le choix de redonner les fonds aux donneurs, ou alors de les
garder en s'imposant d'une marge de 9\%{} fixée par le service. Ce choix est
défini à la création du projet sur Indiegogo.

\paragraph{}
Flattr, qui est, avec Gittip, le seul service facilitant les dons à la
personne, s'attribue une marge qui est la plus importante de ce comparatif.
Contrairement à Gittip il y a un vrai business model, et son fondateur n'en est
d'ailleurs pas à son coup d'essai dans la création d'un service basé sur
l'entraide. En effet, Peter Sunde est aussi l'un des fondateurs du site de
téléchargement (en majorité illégal) The Pirate Bay, dont le business model
était justifié par le fait que le site met à disposition la culture
gratuitement pour tous, avec des revenus publicitaires estimés à 3 millions de
dollars en 2009 (bien que Peter Sunde ait déclaré à cette même période que le
site consommait beaucoup de ressources, et qu'il était en perte).

\paragraph{}
Gittip est donc bien plus respectable éthiquement, ne se basant sur aucun autre
business model que celui du don (via Gittip). Les charges demandées le
prouvent, puisqu'elles sont les même que celles définies par le système de
micropaiements Balance Payments, service utilisé par Gittip pour toute
transaction. Il se trouve dans la tranche basse de ce comparatif, avec des
charges à peine supérieures à celles fixées par Paypal, alors que ce dernier
sert uniquement à transférer de l'argent.

\subsection{la visibilité}

\paragraph{}
Lorsqu'un projet est inscrit sur une plateforme de financement communautaire,
il bénéficie de visibilité supplémentaire grace à son référencement. En effet,
les trois plates-formes utilisées comme exemple, Kickstarter, Flattr et Gittip
disposent d'une liste de membres ou de projets classés par popularité.

\paragraph{}
La plateforme qui rend les projets les plus visibles est sans doute kickstarter
grâce à la sélection en page d'accueil des projets les plus proches
géographiquement ou les plus populaires du moment. Les projets sont classés par
catégories et par date de fin de manière a avantager les projets les plus
proches de l'échéance de financement. Les projets dont l'échéance est passée ne
sont plus mis en évidence.
Gittip peut être intéressant dans la mesure où la communauté reste assez
restreinte pour le moment; il est plus facile de se démarquer. Plus un projet
est financé, plus il occupe une position haute par catégorie.

\paragraph{}
L'utilisation conjuguée de kickstarter et Gittip pourrait se révéler fructueuse
même en cas d'échec de la campagne kickstarter, pour faire connaître le projet.

\paragraph{}
Le modèle pay what you want fonctionne également, dans le cas du Humble Indie
Bundle, pour mettre les jeux les moins connus en avant en les vendant dans la
même offre que des jeux plus populaires, le pic d'intérêt pour les internautes
étant créé par la popularité du site et le fait que les offres, plus
avantageuses que la normale, soient limité dans le temps.

\section{Conclusion}
Les modèles présentés sont plus anciens et ont tous certains avantages suivant
l'usage. Gittip se présente comme une solution alternative au don ponctuel plus
adaptée au développement logiciel que Flattr et également plus transparente.

\paragraph{}
Si Gittip est un modèle prometteur pour la pérennité du logiciel libre il semble
compléter le modèle du Crowdfunding qui a déjà fait ses preuves pour le
lancement de nouveaux projets. Il est envisageable de voir de nouveaux logiciels
open source tirer parti des deux modèles dans les années à venir.
