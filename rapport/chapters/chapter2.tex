\chapter{La problématique de financement de l'open source}

Beacoup de projets libres sont gratuits, mais cela ne signifie pas forcément que
rien n'a été investi : de nombreux modèles de financement innovants permettent
aux développeurs de se consacrer pleinement dans leur projet.

    \section{Les dons}

Le financement de projets par dons est très répandu.
Si ce modèle peut servir d'appoint aux développeurs, il n'est viable que dans
très peu de cas.
    
    \section{donations par micropaiement}

Le modèle du pay per like est une version plus organisée du modèle par dons. Les
dons se font non plus directement au projet mais passent par un oganisme chargé
de répartir équitablement un budget mensuel, défini par le donneur, entre tous
les projets qu'il aura explicitement marqués dans le mois.
\url{flattr.com}, par exemple, permet distruibuer le budget entre les projets
marqués sur \url{github.com}.

    \section{Licence Double}

La licence double est un modèle adopté notamment par MySQL consiste à permettre
l'utilisation de son produit différemment suivant l'usage. Dans le cas de MySQL,
l'utilisation personnelle ou l'inclusion dans un projet sous licence GPL est
gratuite tandis qu'une utilisation profesionnelle est payante.

\paragraph{}
