\chapter{La problématique de financement de l'open source}

Beacoup de projets libres sont gratuits, mais cela ne signifie pas forcément que
rien n'a été investi : de nombreux modèles de financement innovants permettent
aux développeurs de se consacrer pleinement dans leur projet.

    \subsection{Les dons}

Le financement de projets par dons est très répandu.
Si ce modèle peut servir d'appoint aux développeurs, il n'est viable que dans
très peu de cas.
    
    \subsection{donations par micropaiement}

Le modèle du pay per like est une version plus organisée du modèle par dons. Les
dons se font non plus directement au projet mais passent par un oganisme chargé
de répartir équitablement un budget mensuel, défini par le donneur, entre tous
les projets qu'il aura explicitement marqués dans le mois.
\url{flattr.com}, par exemple, permet distruibuer le budget entre les projets
marqués sur \url{github.com}.

    \subsection{Licence Double}

La licence double est un modèle adopté notamment par MySQL consiste à permettre
l'utilisation de son produit différemment suivant l'usage. Dans le cas de MySQL,
l'utilisation personnelle ou l'inclusion dans un projet sous licence GPL est
gratuite tandis qu'une utilisation profesionnelle est payante.

    \subsection{Modèle Redhat}

Redhat distribue son produit gratuitement mais propose un support payant
garantissant la stabilité de son produit. Ce modèle a été adopté par
l'entreprise en 2003, qui depuis cette date est en constante croissance.
C'est un modèle similaire qui régit Canonical.

    \subsection{Modèle Apple}

Le système d'exploiation OSX d'apple est dérivé d'un système ouvert.\\
La stratégie d'apple pour son système d'exploitation OSX a consisté à changer la
licence et apporter des modifications pour ensuite le vendre sur ses machines.

    \subsection{Modèle Bountysource}

\url{bountysource.com} est un site d'annonces de demande de code. Les
utilisateurs peuvent demander des fonctionnalités et se regrouper à plusieurs
sur une même annonce pour augmenter la prime. Ainsi, plus une fonction intéresse
d'utilisateurs, plus la somme proposée pour son implémentation est élevée et
donc susceptible d'intéresser un développeur.

\paragraph{}
