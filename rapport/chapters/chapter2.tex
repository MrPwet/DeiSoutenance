\chapter{La problématique de financement de l'open source}

Beacoup de projets libres sont gratuits, mais cela ne signifie pas forcément
que rien n'a été investi : il existe de nombreux modèles de financement des
projets opensource dont certains initiés par des sociétés comme RedHat ou Apple
qui sont en croissance constante.\\ Le logiciel libre peut donc être rentable à
condition de savoir exploiter le modèle de financement adapté à l'ampleur du
projet.

\section{Quelques modèles alternatifs}

    \subsection{Les dons}

Le financement de projets par dons est très répandu.  Si ce modèle peut servir
d'appoint aux développeurs, il n'est viable que dans très peu de cas.
    
    \subsection{Donations par micropaiement}

Le modèle du pay per like est une version plus organisée du modèle par dons.
Les dons se font non plus directement au projet mais passent par un oganisme
chargé de répartir équitablement un budget mensuel, défini par le donneur,
entre tous les projets qu'il aura explicitement marqués dans le mois.
\url{flattr.com}, par exemple, permet distribuer le budget entre les projets
marqués sur \url{github.com}.\\ Flattr, bien que compatible pour le financement
de projets informatiques est surtout prévu pour récompenser les créateurs de
contenu artistique.
    
    \subsection{Le Crowdfunding}

\paragraph{} En français, on l'appelle "financement collaboratif". Il s'agit
d'un model de financement basé sur la collecte de don. Le leader sur le marché
est le site Kickstarter. Il propose à un utilisateur ou un groupe d'utilisateur
de mettre en ligne l'idée de son projet pour qu'il apparaisse sur le site. Les
visiteurs peuvent alors consulter cette idée et choisir ou non d'investir de
l'argent pou r la voir se développer. Les personnes ayant contribuer au
développement sont très souvent récompensé lors de la fin du projet. Il peuvent
par exemple rencontrer les créateurs du projet qu'ils ont soutenu, obtenir des
goodies\footnote{\textbf{goodies :} produits dérivés faisant référence à un
film, un jeu-vidéo...} ou même pouvoir voter pour l'implémentation de
fonctionnalités.  Ce mode de financement développe l'entraide et donne la
capacité financière afin de mener à bien un projet prometteur.

    \subsection{Licence Double}

La licence double est un modèle adopté notamment par MySQL consiste à permettre
l'utilisation de son produit différemment suivant l'usage. Dans le cas de
MySQL, l'utilisation personnelle ou l'inclusion dans un projet sous licence GPL
est gratuite tandis qu'une utilisation profesionnelle est payante.

    \subsection{Modèle Redhat}

Redhat distribue son produit gratuitement mais propose un support payant
garantissant la stabilité de son produit. Ce modèle a été adopté par
l'entreprise en 2003, qui depuis cette date est en constante croissance.  C'est
un modèle similaire qui régit Canonical.

    \subsection{Modèle Apple}

Le système d'exploiation OSX d'apple est dérivé d'un système ouvert.\\ La
stratégie d'apple pour son système d'exploitation OSX a consisté à changer la
licence et apporter des modifications pour ensuite le vendre sur ses machines.

    \subsection{Modèle Bountysource}

\url{bountysource.com} est un site d'annonces de demande de code. Les
utilisateurs peuvent demander des fonctionnalités et se regrouper à plusieurs
sur une même annonce pour augmenter la prime. Ainsi, plus une fonction
intéresse d'utilisateurs, plus la somme proposée pour son implémentation est
élevée et donc susceptible d'intéresser un développeur.

    \subsection{Modèle Pay What You Want}
    
    \paragraph{} Le modèle Pay What You Want consiste à proposer un produit à
    un utlisateur pour la somme qu'il estime nécessaire de payer. Il peut y
    avoir un seuil minimum à verser ou bien non. Dans le deuxième cas,
    l'utilisateur peut donc acquérir le produit gratuitement de façon légale,
    mais il peut aussi verser un montant qu'il juge adapté au produit de façon
    à récompenser le ou les concepteurs pour leur travail.

    \paragraph{} Ce type de modèle s'est démocratisé en 2007 grâce au groupe de
    musique Radiohead qui a proposé sur son site officiel son nouvel album
    intitulé "in rainbow" afin de lutter contre le piratage. On remarque
    l'utilisation de ce modèle de financement et de ses nombreux dérivés (Pay
    what you wish, Pay what you can etc.) surtout dans le secteur des
    logiciels, de la musique, de la restauration et de l'hotellerie. On peut en
    voir un parfait exemple à l'adresse suivant
    \url{https://www.humblebundle.com/}.


\section{Gittip confronté aux autres modèles}

\subsection{La stabilité des dons}

\paragraph{}
Une part importante de la comparaison entre Gittip et d'autres modèles de
financement concerne la stabilité des revenus. On peut en effet constater qu'un
utilisateur de Gittip qui souhaite apporter sa contribution à un développeur ou
bien à une équipe de développeurs le fait de façon régulière car en
choississant
Gittip, il s'engage à verser une même somme d'argent toutes les semaines.
L'utilisateur peut bien entendu arrêter de verser cette somme dès qu'il le
désire. Dans cette façon de faire, Gittip se démarque des autres modes de
financement qui repose davantage sur des versements ponctuels. 

\paragraph{}
Si on prend l'exemple de Kickstarter, les utilisateurs vont choisir de verser
une certaine somme d'argent de façon ponctuelle et la plupart du temps unique.
Le financement d'un projet lancé sur Kickstarter est donc assez imprévisible.
On peut en effet récolter de très grosses sommes en l'espace de très peu de
temps
(gros dons) ou bien alors une absence totale de don pendant une certaine
période, contrairement à Gittip qui assure un revenu régulier et connu à
l'avance de façon hebdomadaire.

\paragraph{}
La complémentarité des deux modèles précédemment cités peut être très
intéressante. On
imagine en effet parfaitement le financement d'un projet via la plateforme de
Kickstarter (qui requiert une somme souvent importante d'argent) puis le suivi
de ce projet en attribuant un revenu hebdomadaire pour subvenir au besoin
humain du développeur et lui permettre de consacrer son temps à la réalisation
de son projet grâce à Gittip.

\paragraph{}
D'autres modèles en revanche s'éloignent plus de cette problèmatique de
financement. On peut notamment citer le modèle Pay What You Want qui est
beaucoup plus adapté pour vendre le produit fini que pour aider à la conception
de se dernier. De plus l'argent gagné lors de la vente d'un produit ne sert pas
directement à un développeur de ce produit mais il va à la société qui l'a
conçu. De se fait on s'éloigne complètement de Gittip car on s'éloigne de la
notion de salaire. L'argent récolté par la société servira à maintenir le
projet, en créer de nouveaux ou bien à payer les salariés.
