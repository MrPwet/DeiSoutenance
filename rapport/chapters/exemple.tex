\chapter{Exemple}
    \section{Sous-exemple}

\lipsum[1-2]

    \subsection{Sous-sous-exemple}

\begin{itemize}
    \item Item exemple 1
    \item Item exemple 2
    \item Etc\ldots
\end{itemize}

\paragraph{}
J'écris des trucs dans un nouveau paragraphe.\\
Je vais même à la ligne.

Je peux aussi faire comme ça, et ça me fait un alinéa.

\paragraph{Liste à puces}
\begin{itemize}[label=\textbullet]
    \item Item exemple 1
    \item Item exemple 2
\end{itemize}

\paragraph{Tableau}
\begin{center}
\begin{tabular}{|c|c|c|}
    \hline % Permet d'avoir une séparation via un trait
    Colonne 1, ligne 1 & Colonne 2, ligne 1 & Colonne 3, ligne 1\\
    \hline
    Colonne 1, ligne 2 & Colonne 2, ligne 2 & Colonne 3, ligne 2\\
    Colonne 1, ligne 3 & Colonne 2, ligne 3 & Colonne 3, ligne 3\\
    \hline
    \multicolumn{2}{|c|}{Colonne 1 et 2, ligne 4} & Colonne 3, ligne 4\\
    \hline
\end{tabular}
\end{center}
