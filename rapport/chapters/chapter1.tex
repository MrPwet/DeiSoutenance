\chapter{Présentation de Gittip}

Gittip est un système de micro-dons réguliers (par semaine) auprès de
personnes et de projets. Il est possible de donner ou de recevoir si l'on
possède un compte Twitter, Github, ou Bitbucket (les deux derniers étant
des sites très utilisés par les développeurs).

\paragraph{}
Le projet a été lancé le 11 mai 2012 par Chad Whitacre (alias \emph{whit537}),
et a comme mission de "rénover l'économie", en se basant sur la confiance,
la collaboration, le partage, l'ouverture et la transparence.
Il est aussi le créateur de l'open company Gittip.


    \section{Un système de micro-dons libre}

\paragraph{}
Gittip est une application web où un utilisateur peut administrer facilement
ses dons : il peut décider du montant qu'il souhaite donner à une personne
ou à une équipe, et a accès aux profils de tous les utilisateurs.\\
Le code source de Gittip est open source, ce qui signifie qu'il peut être
mit en place sur n'importe quel serveur et peut être modifié librement.
% TODO : reformuler cette phrase, mal dit.
Gittip est accessible depuis l'adresse \url{www.gittip.com} mais aussi sur
n'importe quel site proposant son Gittip.

\paragraph{}
Chaque utilisateur de Gittip peut créer une page de profil où il référence
ses projets et objectifs, dans le but de motiver les donneurs à "investir"
dans notre profil. Il est aussi possible de former des équipes, l'argent
donné étant ensuite réparti équitablement entre chaque membre.\\
Un donneur peut donner au maximum 100\${} à une même personne, l'argent
étant viré tous les mardis. Le montant minimum par don est de 1\textcent, mais
pour minimiser les frais bancaires, l'argent n'est viré que si le montant
total des dons de la semaine dépasse 10\${}, sinon les dons sont reportés à
la semaine suivante.\\
Gittip étant pensé pour du financement sur le long terme, on ne peut faire
de dons spontanés et uniques.

\paragraph{}
Comme le projet est libre, la philosophie du projet pourra être différente
suivant qui met en place Gittip (car le projet peut être modifié par
quiconque le souhaite). C'est pourquoi \url{Gittip.com} a souhaité fonder
sa propre compagnie, que Chad Whitacre a lui même qualifié d'"open company",
la Gittip LLC.


    \section{OpenCompany}

Fondée par Chad Whitacre en 2002 sous le nom de \emph{Zeta Web Design}
puis renommée en \emph{Gittip, LLC} en février 2012, et elle est fondée sur
le principe de Gittip : les employés ne sont pas directement rémunérés
par la société, mais par quiconque souhaitant donner à l'équipe "Gittip"
depuis Gittip. \emph{Gittip, LLC} ne fait pas de profit, les seuls frais
demandés aux donneurs sont de $30$\textcent $\, + \, 2,9\%{}$ pour les frais
bancaires engendrés par les transactions.

\paragraph{}
Tout contributeur au projet Gittip est membre de la \emph{Gittip, LLC},
et tout utilisateur de \url{gittip.com} en est le client. L'objectif de cette
Open Company est d'être le plus transparent possible, tant que cela ne nuit
pas à la vie privée des utilisateurs ou des employés : les communications
entre employés se font dans la mesure du possible via des moyens où la mise
à disposition pour tous est simple (chat, vidéos conférences envoyées ensuite
sur Youtube, etc\ldots), toute fraude est communiquée et toute modification
est visible par la communauté.

% Partie non terminée

    \subsection{Conditions}

    \section{Spécificité}
