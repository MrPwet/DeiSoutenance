\chapter{Présentation de Gittip}

Gittip est un système de micro-dons réguliers (par semaine) auprès de personnes
et de projets. Il est possible de donner ou de recevoir si l'on possède un
compte Twitter, Github, ou Bitbucket (les deux derniers étant des sites très
utilisés par les développeurs).

\paragraph{}
Le projet a été lancé le 11 mai 2012 par Chad Whitacre (alias \emph{whit537}),
et a comme mission de ``rénover l'économie'', en se basant sur la confiance, la
collaboration, le partage, l'ouverture et la transparence. Il est aussi le
créateur de l'open company \emph{Gittip LLC}. Le site \url{www.gittip.com}
compte, début novembre 2013, 22 000 utilisateurs, dont 2 000 utilisateurs
actifs (qui donnent ou reçoivent de l'argent via Gittip chaque semaine), pour
un total de 7 300\${} transférés chaque semaine.


    \section{Un système de micro-dons libre}

    \subsection{Aspects communautaires}

Chaque utilisateur de Gittip peut créer une page de profil où il référence ses
projets et objectifs, dans le but de motiver des donneurs à ``investir'' dans
son profil. Il est aussi possible de former des équipes, l'argent donné à
l'équipe est ensuite réparti entre chaque membre.

\paragraph{}
Chaque membre d'une équipe définit la somme qu'il souhaite recevoir sur le
montant total donné à l'équipe. L'ordre de priorité est établi en fonction de
l'ordre d'inscription des membres à l'équipe : les membres les plus récemment
admis dans l'équipe sont prioritaires. Pour permettre aux nouveaux membres de
l'équipe de faire leurs preuves, ils ne peuvent pas dépasser un certain montant,
la première semaine fixé à 1\textcent{}, la deuxième semaine à 1\${},
puis cette limite est doublée chaque semaine.

\paragraph{}
Les montants définis par chaque membre sont publics. Ainsi, un montant demandé
qui est estimé trop élevé sera critiqué par l'équipe. Ce système de
transparence est un moyen simple d'éviter les abus en confiant à l'équipe la
tâche de gérer ses membres. Si la somme totale distribuée aux membres est
inférieure à la somme totale donnée à l'équipe, l'argent est mis de côté pour
la semaine suivante, ou alors viré sur le compte bancaire de l'équipe au cas où
un compte lui a été assigné.

\paragraph{}
Rencontrer d'autres personnes pour former une équipe permet aux utilisateurs de
Gittip de s'investir sur un projet plus important, qui permettra à l'équipe de
rapporter potentiellement plus de dons. Les équipes sont limitées à 150
membres, soit le nombre de Dunbar, qui \emph{`` est le nombre maximum
d'amis avec lesquels une personne peut entretenir une relation stable à un
moment donné de sa vie. '' (wikipedia)}, pour maintenir une bonne communication
entre les équipes.

\paragraph{}
Bien que le système d'équipes ait été mis en place, il ne doit pas interférer
avec le but premier de Gittip qui est de s'intéresser à la personne, non pas
comme beaucoup d'autres modèles qui se concentrent uniquement sur le projet.
S'il est bénéfique de rejoindre une équipe pour gagner en visibilité, beaucoup
d'utilisateurs gagnent finalement plus grâce aux dons leur étant adressés
directement plutôt que grâce aux gains des équipes dont ils font partie.

    \subsection{Le profit n'est pas un but}

Gittip est une application web où un utilisateur peut gérer facilement ses dons
: il peut décider du montant qu'il souhaite donner à une personne ou une
équipe, et a accès aux profils de tous les utilisateurs. Un donneur peut verser
au maximum 100\${} à une même personne, l'argent étant viré tous les mardis. Le
montant minimum par don est de 1\textcent, mais pour minimiser les frais
bancaires, l'argent n'est viré que si le montant total des dons de la semaine
dépasse 10\${}, sinon les dons sont reportés à la semaine suivante.  Gittip
étant pensé pour du financement sur le long terme, on ne peut faire de dons
spontanés et uniques. Les comptes bancaires sont liés au compte Gittip via le
système de paiement open source \emph{Balanced Payments}.

    \subsection{Liberté symbolisée par l'aspect open source}

Gittip est un projet open source, ce qui signifie qu'il peut être mis en place
sur n'importe quel serveur et peut être modifié librement.  Il est accessible
depuis l'adresse \url{www.gittip.com} mais aussi sur n'importe quel site
proposant son Gittip. Même s'il est open source, le site n'a pas
grand intérêt à être forké\footnote{\textbf{Fork :} clone d'un projet open
source pour être modifié et gardé séparé du projet initial.}, car son intérêt
est sa base d'utilisateurs, et mettre en place le projet sur un autre serveur
équivaut à se priver d'une partie de cette communauté.

\paragraph{}
Comme le projet est libre, sa philosophie dépendra donc de qui héberge Gittip.
C'est pourquoi Chad Whitacre a souhaité fonder sa propre compagnie, qu'il a lui
même qualifié d'\emph{Open Company} : la \emph{Gittip LLC}.


    \section{L'open company \emph{Gittip LLC}}

Fondée par Chad Whitacre en 2002 sous le nom de \emph{Zeta Web Design} puis
renommée en \emph{Gittip LLC} en février 2012, et elle est basée sur le
principe de Gittip : les employés ne sont pas directement rémunérés par la
société, mais par quiconque souhaitant donner à l'équipe \emph{Gittip} depuis
Gittip. \emph{Gittip LLC} ne fait pas de profit, les seuls frais demandés aux
donneurs, dépendant de la somme donnée, sont de $30$\textcent $\, + \, 2,9\%{}$
pour les frais bancaires engendrés par les transactions. Elle compte,
début novembre 2013, 176 employés.

\paragraph{}
L'open company comme définie par Chad Whitacre se base sur trois idées
principales : partager autant que possible, éviter au maximum les coûts et ne
pas rémunérer directement les employés.

    \subsection{Politique basée sur la transparence}

Le type d'open company fut inventé par Alexander Stigsen en mars 2009 pour sa
société \emph{E Text Editor}. Malheureusement, peu d'informations sont
disponibles aujourd'hui à propos de cette première open company; le site
officiel étant inaccessible au même titre que ses archives, sa page wikipedia
étant incomplète et la société ayant eu une faible notoriété, peu d'autres sources
d'informations la citent. L'idée d'A. Stigsen était de bâtir une entreprise en
trois étapes : publier les sources, construire un \emph{Trust Metric} puis
rémunérer les participants.

\paragraph{}
Le Trust Metric, ou indicateur de confiance en français, était prévu comme
étant un algorithme permettant de répartir les gains de l'entreprise
de façon équitable pour chaque employé, suivant le travail fourni pour
le projet : un employé qui s'investissait plus qu'un autre aurait un salaire
plus important.

\paragraph{}
La \emph{Gittip LLC} est donc la deuxième Open Company créée, bien que la
philosophie de l'open company, légèrement remaniée par C. Whitacre, se base sur
le modèle de l'\emph{Open business}. Ce modèle n'a jamais été la base d'une
société, mais plutôt repris par les Fondations (ou associations à but non
lucratif), telles que la Mozilla Foundation. Le problème de ces associations
étant qu'elles sont soumises à une législation dépendant du pays, et donc à
certaines contraintes. Ces contraintes sont différentes dans le cas d'une Open
Company; ce point sera développé dans \hyperref[chapter3]{le troisième chapitre
de ce rapport}.

\paragraph{}
Tout contributeur au projet Gittip est un employé de la \emph{Gittip LLC}, et
tout utilisateur de \url{gittip.com} en est le client. L'objectif de cette Open
Company est d'être la plus transparente possible, tant que cela ne nuit pas à
la vie privée des utilisateurs ou des employés : les communications entre
employés se font dans la mesure du possible via des plates-formes consultables de
tous (chat, visioconférences Youtube, etc.), toute fraude est communiquée
et toute modification est visible par la communauté.
De plus, tout point de la politique de Gittip est documenté, librement accessible
et tout le monde a la possibilité de commenter ces points et de partager les
problèmes ressentis dans le fonctionnement du site, du modèle économique ou de
la politique de l'entreprise.

    \subsection{Société sans hiérarchie}

\emph{Gittip LLC} n'a pas de hiérarchie, tous les employés sont membres de
l'équipe Gittip sans statut particulier. Cette absence de hiérarchie peut
paraître candide, mais il existe des entreprises dans laquelle cela a déjà été
mis en place comme chez le géant de l'industrie vidéoludique Valve,
dont la valeur
était estimée à 2,5 Milliards de dollars en 2012 par le New York Times
(l'entreprise n'étant pas cotée en bourse, il n'y a pas de valeur précise
disponible).

\paragraph{}
Valve a géré cette non hiérarchie en fixant des salaires assez bas pour les
employés, et en accordant des bonus pouvant atteindre 5 à 10 fois le montant du
salaire initial, qu'il est possible de décrocher via de bons résultats suite
aux évaluations établies par les autres collègues. Les recrutements,
licenciement et décisions par rapport à l'entreprise se font par le biais de
réunions, et non selon le seul avis des ressources humaines et du fondateur :
Gabe Newell.

\paragraph{}
Gittip se base sur un fonctionnement similaire, où tout dépend de la
confiance et la conversation : un employé n'ayant que très peu participé au
projet et étant le membre demandant le plus dans l'équipe sera critiqué par ses
collègues, et pourra se voir exclu si aucune entente n'est possible. Un employé
ne doit donc pas décevoir ses collègues en fournissant une charge de travail
trop légère par rapport au salaire qu'il souhaite avoir. Il n'y a
finalement pas de hiérarchie définie mais plutot une dépendance mutuelle
et vertueuse.


    \section{Conclusion}

Gittip est aussi bien un modèle de financement novateur qu'une entreprise tout
aussi innovante. Les argumentations des parties précédentes ont montré que la
\emph{Gittip LLC} peut survivre, malgré certains choix allant à contre courant
des modèles communs d'entreprise, comme son absence de hiérarchie. Sa réussite
pourra potentiellement jouer un rôle dans la notoriété du statut d'open
company, ainsi pourra-t-on voir d'autres sociétés de ce genre, ce qui
contribuerait aussi à la propagation du modèle de financements Gittip.
