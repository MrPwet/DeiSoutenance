\chapter{Présentation de Gittip}

Gittip est un système de micro-dons réguliers (par semaine) auprès de personnes
et de projets. Il est possible de donner ou de recevoir si l'on possède un
compte Twitter, Github, ou Bitbucket (les deux derniers étant des sites très
utilisés par les développeurs).

\paragraph{}
Le projet a été lancé le 11 mai 2012 par Chad Whitacre (alias \emph{whit537}),
et a comme mission de "rénover l'économie", en se basant sur la confiance, la
collaboration, le partage, l'ouverture et la transparence. Il est aussi le
créateur de l'open company Gittip.


    \section{Un système de micro-dons libre}

    \subsection{Aspects communautaires}

Chaque utilisateur de Gittip peut créer une page de profil où il référence ses
projets et objectifs, dans le but de motiver des donneurs à "investir" dans
notre profil. Il est aussi possible de former des équipes, l'argent donné à
l'équipe est ensuite réparti équitablement entre chaque membre.

\paragraph{}
Rencontrer d'autres personnes pour former une équipe permet aux utilisateurs de
Gittip de s'investir sur un projet plus important, qui permettra à l'équipe de
rapporter potentiellement plus de dons. Les équipes sont limitées à 150
membres, soit le nombre de Dunbar, qui est `` \emph{est le nombre maximum
d'amis avec lesquels une personne peut entretenir une relation stable à un
moment donné de sa vie. (wikipedia)} '', pour maintenir une bonne communication
entre les équipes.
\end{quote}

    \subsection{Le profit n'est pas un but}

Gittip est une application web où un utilisateur peut gérer
facilement ses dons : il peut décider du montant qu'il souhaite donner à une
personne ou à une équipe, et a accès aux profils de tous les utilisateurs.\\ Un
donneur peut verser au maximum 100\${} à une même personne, l'argent étant viré
tous les mardis. Le montant minimum par don est de 1\textcent, mais pour
minimiser les frais bancaires, l'argent n'est viré que si le montant total des
dons de la semaine dépasse 10\${}, sinon les dons sont reportés à la semaine
suivante.  Gittip étant pensé pour du financement sur le long terme, on ne peut
faire de dons spontanés et uniques.

    \subsection{Liberté symbolisée par l'aspect opensource}

Gittip est un projet open source, ce qui signifie qu'il peut être
mit en place sur n'importe quel serveur et peut être modifié librement.
Il est accessible depuis l'adresse \url{www.gittip.com} mais aussi sur
n'importe quel site proposant son Gittip. Même si il est opensource,
\url{www.gittip.com} n'a pas grand intérêt à être forké\footnote{\textbf{Fork
:} clone d'un projet open source pour être modifié et gardé séparé du
projet initial.}, car son intérêt est sa base d'utilisateurs et mettre en place
le projet sur un autre serveur équivaut à se priver d'une partie de cette base
d'utilisateurs.

\paragraph{}
Comme le projet est libre, sa philosophie dépendra donc de qui héberge Gittip
(car le projet peut être modifié par quiconque le souhaite). C'est pourquoi
\url{gittip.com} a souhaité fonder sa propre compagnie, que Chad Whitacre a lui
même qualifié d'``Open Company'', la Gittip LLC.


    \section{OpenCompany}

Fondée par Chad Whitacre en 2002 sous le nom de \emph{Zeta Web Design} puis
renommée en \emph{Gittip, LLC} en février 2012, et elle est basée sur le
principe de Gittip : les employés ne sont pas directement rémunérés par la
société, mais par quiconque souhaitant donner à l'équipe \emph{Gittip} depuis
Gittip. \emph{Gittip, LLC} ne fait pas de profit, les seuls frais demandés aux
donneurs, dépendant de la somme donnée, sont de $30$\textcent $\, + \, 2,9\%{}$
pour les frais bancaires engendrés par les transactions.

    \subsection{Politique basée sur la transparence}

La \emph{Gittip, LLC} est la première Open Company créée, le type ayant été
inventé par Chad Whitacre, et se base sur modèle de l'Open business. Ce modèle
n'a jamais été la base d'une société, mais plutôt reprit par les Fondations (ou
associations à but non lucratif), telles que la Mozilla Foundation. Le problème
de ces associations étant qu'elles sont soumises à une législation dépendant du
pays, et donc à certaines contraintes. Ces contraintes sont différentes dans le
cas d'une Open Company, ce point sera développé dans \hyperref[chapter3]{le
troisième chapitre de ce rapport}.

\paragraph{}
Tout contributeur au projet Gittip est un employé de la \emph{Gittip, LLC}, et
tout utilisateur de \url{gittip.com} en est le client. L'objectif de cette Open
Company est d'être la plus transparente possible, tant que cela ne nuit pas à
la vie privée des utilisateurs ou des employés : les communications entre
employés se font dans la mesure du possible via des moyens où la mise à
disposition pour tous est simple (chat, vidéos conférences envoyées ensuite sur
Youtube, etc\ldots), toute fraude est communiquée et toute modification est
visible par la communauté.
