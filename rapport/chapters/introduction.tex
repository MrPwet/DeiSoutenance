\chapter{Introduction}

Le but de ce rapport est de présenter le système de financement \emph{Gittip},
en se basant sur la problématique suivante :

\begin{center}
\textbf{Gittip a t'il une chance de devenir un système de
financements reconnu et répandu, en France et dans le monde ?}
\end{center}

\paragraph{}
En tant qu'étudiants, il arrive parfois que des idées de projets fleurissent
mais ne dépassent pas le stade d'idée par manque de budget.\\
Il est donc intéressant d'être informé des différents moyens de financement
possibles avant de se lancer dans un projet, c'est pourquoi nous avons
choisi ce sujet.

\paragraph{}
% L'idée : tout n'est pas tout beau tout rose
On s'intéressera aux différents problèmes que rencontrent les moyens
de financements à leurs débuts, à savoir des problèmes juridiques
(imposition, etc\ldots), mais aussi aux problèmes de mise en place et des
différents vices qui peuvent être rencontrés.

\paragraph{}
Le dossier se concentre particulièrement sur le système Gittip car il
est très ambitieux dans sa mission, et c'est aussi un moyen de financement
très jeune et encore méconnu du grand public (contrairement aux grands sites
de crowdundings comme KickStarter ou Indiegogo).\\
Le principe d'un moyen de financement opensource avec la création d'une
entreprise libre est peu commun et suscite certaines questions quand à sa
mise en place.

\paragraph{}
La présentation de Gittip sera la premier point à être développé, de
façon à pouvoir le comparer ensuite à d'autres moyens de financements
sur le plan pratique, ce qui sera notre deuxième partie,
que sur le plan légal, qui construira notre troisième partie, pour pouvoir
approximer une réponse sur la possible mise en place de Gittip en conclusion
de ce rapport.
