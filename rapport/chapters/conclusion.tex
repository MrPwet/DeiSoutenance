\chapter*{Conclusion}
\addcontentsline{toc}{chapter}{Conclusion}


% 1 Le modèle est intéressant et peut offrir une solution pour financer l'open source, de la façon khan academy
% 2 Le développement de la solution n'est pas encore optimal et comme toutes les grandes problématiques, il faudra du temps
% 3 La situation française rend l'accessibilité à ce modèle assez compliqué et il est à ésperer que les choses changeront

\paragraph{}
Attendu les éléments que nous avons mis en avant, nous pouvons conclure notre 
propos avec les points suivants.

\paragraph{}
Nous avons constaté le changement de point de vue qu'offre gittip, du projet 
dans les modèles de financement historiques vers l'individu.
Nous pensons que ce changement est pour le mieux et va dans la bonne 
direction.
En effet, rémunérer les développeurs directement plutôt que de remplir les 
caisses des projets sur lesquels ils travaillent permet d'offrir à ces 
développeurs une motivation pour continuer à travailler comme ils le font, et 
donc nous assure la pérrénité de ces projets.
Il sera bien entendu nécessaire de faire en sorte que les finances propres des 
projets permettent de payer pour leurs infrastructures, comme par exemple des 
serveurs ou de la bande passante, mais là encore, gittip permet cela par le 
mécanisme des teams.

\paragraph{}
Nous exprimerons également notre enthousiasme sur la pratique de l'entreprise 
Khan Academy de réserver un budget hebdomadaire par employé qui est donné par 
le biais de gittip à une personne du choix de l'employé en question.
Nous pensons qu'il s'agit d'une excellente façon pour une entreprise de 
participer au financement du logiciel libre.

\paragraph{}
Ensuite, malgré cet enthousiasme, force est de constater que la solution 
proposée laisse encore quelques points d'ombre. Par exemple, le montant gagné 
par un développeur très visible mais peu efficace sera presque toujours plus 
important que celui gagné par un développeur très efficace mais discret.
Il ne s'agit donc toujours pas du modèle parfait, et comme pour toute grande 
problématique, la résolution de celle du financement du logiciel libre prendra 
encore du temps.

\paragraph{}
Enfin, nous constatons que de notre côté de l'Atlantique, le système n'est pas 
aussi viable qu'il pourrait l'être, à cause notamment de la faible culture du 
don en France, du relatif vide juridique autour de la question, et des 
difficultés fiscales rencontrées du fait de celui-ci. Il ne faut cependant pas 
se montrer pessimiste : ce genre de barrières ont tendance à évoluer et si le 
modèle de gittip rencontre un grand succès dans d'autres pays, il n'est à 
point douter que la France, par sa position dominante dans le logiciel libre, 
saura s'y adapter.
