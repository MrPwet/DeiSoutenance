%Ceci est encore une ébauche
\chapter{La problématique de la légalité du financement en France}\label{chapter3}
    \section{Introduction}
        \paragraph{}
            % Rappel des éléments précédents
            Nous avons jusqu'à présent étudié l'entrerpise et le site
            internet Gittip, ainsi que certains modèles de financements
            proposés pour pallier au problème du financement du logiciel
            libre, nous allons à présent nous intéresser plus en détail
            à l'aspect légal, et en particulier au volet fiscal, en relation
            avec l'utilisation des services proposés par gittip.
        \paragraph{}
            % Faire apparaître les 3 intervenants : le donneur, le receveur et gittip
            Avant tout, nous devrons distinguer les trois acteurs en jeu
            lors de l'utilisation du service de micro dons, et leurs rôles
            respectifs dans l'opération. 
            % Introduire le volet de fiscalité américaine
            Ensuite, gittip étant une société enregistrée aux États-unis
            il est nécessaire de se pencher brièvement sur son fonctionnement
            dans les conditions pour lesquelles son modèle a été pensé,
            c'est à dire le droit américain.
            % Droit français et conclusion
            Finalement, nous rentrerons dans le vif du sujet en détaillant les
            implications au niveau du droit fiscal français quant à
            l'utilisation des services proposés, et dresserons alors un bilan
            d'aptitude à l'utilisation de gittip en France

    \section{Généralités}
        \subsection{Acteurs}
            \paragraph{}
                Un don récurrent sur Gittip fait intervenir trois acteurs
                principaux.
                Tout d'abord le \emph{donateur}, de qui l'argent émane.
                Il s'agit le plus souvent d'un particulier, même si dans
                certains cas, il peut s'agir d'une organisation.
                Lorsque qu'il émet un don récurrent, gittip,
                l'\emph{intermédiaire} détermine si 
                le donateur possède suffisamment d'argent sur son compte,
                débite sa carte de crédit si nécessaire et procède au
                virement vers le compte gittip du \emph{donataire}. Ce dernier
                reçoit l'argent du donateur et peut ensuite le virer sur son
                compte bancaire ou l'utiliser pour faire lui-même un don.
                Nous utiliserons les concepts de donateur, intermédiaire et
                donataires tout au long de ce chapitre. 
    \section{Une entreprise américaine}
            % http://www.irs.gov/Businesses/Small-Businesses-%26-Self-Employed/Frequently-Asked-Questions-on-Gift-Taxes
            % http://www.irs.gov/pub/irs-prior/p950--2011.pdf
            % https://en.wikipedia.org/wiki/Commissioner_v._Duberstein
            %
            % Elements de droit américain. TL;DR : un donateur peut donner ce
            % qu'il veut à qui il veut tant que c'est sous 13.000$/an
            % 
            % Le donataire a rien à faire parce qu'à priori, c'est au donateur
            % au'incombent les taxes
            % 
            % Gittip doit envoyer un 1099K à l'IRS pour tout donateur au dessus de 20.000$/ans ET 200 transactions/ans
        \paragraph{}
            Les États-Unis sont une nation possédant une très forte cutlure
            du don. En effet, en favorisant l'autonomie vis à vis de leurs
            institutions gouvernantes, ce pays se base sur un grand nombre
            d'organisations de charités ayant pour but de donner aux citoyens
            les outils nécessaures à combattre les problèmes sociaux de leur
            société sans s'en remettre à un pouvoir politique, comme on
            pourrait le voir en France.
            D'autre part, le don est égallement un geste du quotidien :
            le tip (pourboire) pour rémunérer le service de divers
            professionnels tels que les chauffeurs de taxis, les serveurs ou
            encore les coiffeurs, n'est pas optionnel mais constitue une 
            obligation sociale.
            Dans ce contexte, la législation fiscale américaine est
            complète et permissive à ce sujet.
        \subsection{Donateur}
            Bien que l'IRS prévoit des taxes sur les dons, elles ne
            s'appliquent qu'à une fraction des cas de figures.
            Particulièrement, elles ne s'appliquent pas aux dons dont les
            donataires sont des parents du donateur, des organismes de
            charité ou des partis politiques, ni aux dons destinés à payer
            pour des dépenses médicales ou des frais d'éducation.
            De plus, pour les dons assujétis à la taxe sur les dons, il
            existe un large abattement annuel, fixé à 13000\dollar soit plus
            de 9500\euro pour l'année 2012. Si, malgré cela, les sommes
            données venaient à dépasser cet abattement, il reviendrait au
            donateur de s'acquitter de cette taxe.
        \subsection{Donataire}
            Comme nous l'avons vu, le poid fiscal repose, aux États-Unis, 
            sur les épaule du donateur. Cela est totalement transparent
            pour le donataire qui peut donc recevoir jusqu'à 13000\dollar
            par donateur par an sans que l'IRS ne s'intéresse à son cas.
        \subsection{Institution bancaire}
    \section{Utilisée par des français}
            % TL;DR beaucoup trop à dire là, voir mes notes
        \subsection{Donateur}
        \subsection{Donataire}
        \subsection{Institution bancaire}
    \section{Conclusion}
