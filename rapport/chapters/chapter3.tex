%Ceci est encore une ébauche
\chapter{La problématique de la légalité du financement en France}\label{chapter3}
    \section{Introduction}
        \paragraph{}
            % Rappel des éléments précédents
            Nous avons jusqu'à présent étudié l'entrerpise et le site
            internet Gittip, ainsi que certains modèles de financements
            proposés pour pallier au problème du financement du logiciel
            libre, nous allons à présent nous intéresser plus en détail
            à l'aspect légal, et en particulier au volet fiscal, en relation
            avec l'utilisation des services proposés par gittip.
        \paragraph{}
            % Faire apparaître les 3 intervenants : le donneur, le receveur et gittip
            Avant tout, nous devrons distinguer les trois acteurs en jeu
            lors de l'utilisation du service de micro dons, et leurs rôles
            respectifs dans l'opération. 
            % Introduire le volet de fiscalité américaine
            Ensuite, gittip étant une société enregistrée aux États-unis
            il est nécessaire de se pencher brièvement sur son fonctionnement
            dans les conditions pour lesquelles son modèle a été pensé,
            c'est à dire le droit américain.
            % Droit français et conclusion
            Finalement, nous rentrerons dans le vif du sujet en détaillant les
            implications au niveau du droit fiscal français quant à
            l'utilisation des services proposés, et dresserons alors un bilan
            d'aptitude à l'utilisation de gittip en France

    \section{Généralités}
        \subsection{Acteurs}
            \paragraph{}
                Un don récurrent sur Gittip fait intervenir trois acteurs
                principaux.
                Tout d'abord le \emph{donateur}, de qui l'argent émane.
                Il s'agit le plus souvent d'un particulier, même si dans
                certains cas, il peut s'agir d'une organisation.
                Lorsque qu'il émet un don récurrent, \emph{gittip} détermine si 
                le donnateur possède suffisamment d'argent sur son compte,
                débite sa carte de crédit si nécessaire et procède au
                virement vers le compte gittip du \emph{donataire}.
    \section{Une entreprise américaine}
        \subsection{Donnateur}
        \subsection{Donnataire}
        \subsection{Institution bancaire}
    \section{Utilisée par des français}
        \subsection{Donnateur}
        \subsection{Donnataire}
        \subsection{Institution bancaire}
    \section{Conclusion}
