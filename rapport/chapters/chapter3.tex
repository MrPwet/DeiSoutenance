%Ceci est encore une ébauche
\chapter{Fiscalité}\label{chapter3}
        \paragraph{}
            % Rappel des éléments précédents
            Nous avons jusqu'à présent étudié l'entreprise et le site
            internet Gittip, ainsi que certains modèles de financements
            proposés pour pallier au problème du financement du logiciel
            libre, nous allons à présent nous intéresser plus en détail
            à l'aspect légal, et en particulier au volet fiscal, en relation
            avec l'utilisation des services proposés par Gittip.
        \paragraph{}
            % Faire apparaître les 3 intervenants : le donneur, le receveur et Gittip
            Avant tout, nous devrons distinguer les trois acteurs en jeu
            lors de l'utilisation du service de micro dons, et leurs rôles
            respectifs dans l'opération.
            % Introduire le volet de fiscalité américaine
            Ensuite, Gittip étant une société enregistrée aux États-unis
            il est nécessaire de se pencher brièvement sur son fonctionnement
            dans les conditions pour lesquelles son modèle a été pensé,
            c'est à dire le droit américain.
            % Droit français et conclusion
            Finalement, nous rentrerons dans le vif du sujet en détaillant les
            implications au niveau du droit fiscal français quant à
            l'utilisation des services proposés, et dresserons alors un bilan
            d'aptitude à l'utilisation de Gittip en France

    \section{Généralités}
        \subsection{Acteurs}
            Un don récurrent sur Gittip fait intervenir trois acteurs
            principaux.
            Tout d'abord le \emph{donateur}, de qui l'argent émane.
            Il s'agit le plus souvent d'un particulier, même si dans
            certains cas, il peut s'agir d'une organisation.
            Lorsque qu'il émet un don récurrent, Gittip,
            l'\emph{intermédiaire}, détermine si
            le donateur possède suffisamment d'argent sur son compte,
            débite sa carte de crédit si nécessaire et procède au
            virement vers le compte Gittip du \emph{donataire}. Ce dernier
            reçoit l'argent du donateur et peut ensuite le virer sur son
            compte bancaire ou l'utiliser pour faire lui-même un don.
            Nous utiliserons les concepts de donateur, intermédiaire et
            donataires tout au long de ce chapitre.
    \section{Aux États-Unis}
            % http://www.irs.gov/Businesses/Small-Businesses-%26-Self-Employed/Frequently-Asked-Questions-on-Gift-Taxes
            % http://www.irs.gov/pub/irs-prior/p950--2011.pdf
            % https://en.wikipedia.org/wiki/Commissioner_v._Duberstein
            %
            % Éléments de droit américain. TL;DR : un donateur peut donner ce
            % qu'il veut à qui il veut tant que c'est sous 13.000$/an
            %
            % Le donataire a rien à faire parce qu'à priori, c'est au donateur
            % au'incombent les taxes
            %
            % Gittip doit envoyer un 1099K à l'IRS pour tout donateur au dessus de 20.000$/ans ET 200 transactions/ans
        \paragraph{}
            Les États-Unis sont une nation possédant une très forte culture
            du don. En effet, en favorisant l'autonomie vis à vis de leurs
            institutions gouvernantes, ce pays se base sur un grand nombre
            d'organisations de charités ayant pour but de donner aux citoyens
            les outils nécessaires à combattre les problèmes sociaux de leur
            société sans s'en remettre à un pouvoir politique, comme on
            pourrait le voir en France.
            D'autre part, le don est également un geste du quotidien :
            le tip (pourboire) pour rémunérer le service de divers
            professionnels tels que les chauffeurs de taxis, les serveurs ou
            encore les coiffeurs, n'est pas optionnel mais constitue une
            obligation sociale.
            Dans ce contexte, la législation fiscale américaine est
            complète et permissive à ce sujet.
        \subsection{Donateur}
            Bien que l'IRS prévoie des taxes sur les dons, elles ne
            s'appliquent qu'à une fraction des cas de figures.
            Particulièrement, elles ne s'appliquent pas aux dons dont les
            donataires sont des parents du donateur, des organismes de
            charité ou des partis politiques, ni aux dons destinés à payer
            pour des dépenses médicales ou des frais d'éducation.
            De plus, pour les dons assujetis à la taxe sur les dons, il
            existe un large abattement annuel, fixé à 13000\$ soit plus
            de 9500\euro{} pour l'année 2012. Si, malgré cela, les sommes
            données venaient à dépasser cet abattement, il reviendrait au
            donateur de s'acquitter de cette taxe.
        \subsection{Donataire}
            Comme nous l'avons vu, le poids fiscal repose, aux États-Unis,
            sur les épaule du donateur. Cela est totalement transparent
            pour le donataire qui peut donc recevoir jusqu'à 13000\$
            par donateur par an sans que l'IRS ne s'intéresse à son cas.
        \subsection{Intermédiaire}
            L'intermédiaire n'est que très peu concerné par les
            considérations fiscales que ce soit du donateur ou du donataire.
            Il doit cependant déclarer à l'IRS les informations de
            tout donateur ayant donné au moins 20.000\$ pour une année en
            ayant fait au moins 200 transactions.
    \section{En France}
            % Deux grandes classes de dons : dons et présents d'usage
            % Dons :
            %   * droit d'enregistrement à payer
            %   * taxés jusqu' 60%
            %   * abattements seulement pour la famille et les handicapés
            %   * abattements renouvelés que tous les 15 ans
            %
            % Présents d'usage
            %   * doivent être modestes vis-à-vis du patrimoine
            %   * doivent être offerts à des occasions particulières (noel, mariage)
            %
            % Ici, vu que c'est récurrent, c'est pas vraiment un présent d'usage
            % -> problème
            %
            % Aussi c'est le Donataire
        \paragraph{}
            La situation française est bien différente. En effet, la
            législation prévoit deux grandes catégories de dons. D'une
            part, les présents d'usage, d'autre part les donations
            telles que les dons manuels. Nous verrons que les dons au sens
            de Gittip ne rentrent dans aucune de ces deux catégories,
            compliquant la tâche au donneur comme au receveur.

        \subsection{Nature fiscale du don}

            \subsubsection{Les présents d'usages}
                Les présents d'usage sont des dons non-imposables offerts à des
                occasions particulières telles qu'un mariage ou qu'une fête
                religieuse.
                Afin d'être reconnu comme tel, le présent d'usage doit être
                inférieur à une valeur proportionnée au patrimoine du donateur.
                Nous pouvons donc observer que le don au sens de Gittip ne
                rentre pas dans cette catégorie, notamment à cause de
                l'incompatibilité entre la récurrence du don et la nécessité de
                justifier le présent d'usage par une événement familial.

            \subsubsection{Les donations}
                Les donations sont n'importe quel don qui ne sont pas des présents
                d'usage. Ils portent généralement sur des sommes plus importantes
                que ceux-ci et sont soumis au droit d'enregistrement, un impôt
                dont le taux dépend du degré de proximité familiale entre le
                donateur et le donataire.
                Il existe des abattements sur cet impôt, se renouvelant tous
                les 15 ans et dépendant eux aussi de la proximité entre le
                donneur et le receveur. Cet impôt est payable par le donataire.
                Ainsi, dans le cas des dons dans le sens de Gittip, la donation
                n'offre pas non plus de cadre juridique satisfaisant, car le
                donateur étant anonyme, il est difficile d'établir un lien ou
                l'absence d'un lien de parenté entre ce dernier et le donataire.

            \subsubsection{Le revenu}
                Attendu le caractère récurrent des dons transférés par
                le biais de Gittip, ces derniers seront considérés par
                l'administration comme un revenu.
                Ce revenu ne peut être considéré comme un salaire, par
                l'absence d'un lien de subordination, ni comme la
                rémunération d'une prestation, aucune prestation n'ayant
                justifié l'entrée d'argent, ni de bénéfices commerciaux,
                les projets sur lesquels le donataire travaille étant libres
                et à priori non commerciaux.
                Restent les bénéfices non commerciaux, catégorie de
                revenus généralement utilisée lorsque aucune autre
                catégorie ne s'applique.
                Cette catégorie est la meilleure solution pour déclarer les
                dons reçus sur Gittip, mais ne constituent pas le modèle
                parfait, en effet, il existe autour de la question un vide
                juridique qui empêche de trouver des réponses claires et
                définitive à la question de la nature de ce type de revenus.

        \subsection{Donateur}
            Bien que la situation ne soit pas des plus évidente pour les dons
            en France, le donateur bénéficie d'une tranquillité relative, ne
            devant pas déclarer ses dons lui même. Il lui suffira donc de
            donner le montant désiré sans se soucier du côté fiscal. Cela
            pose bien sûr un certain nombre de questions quant à la légalité
            de ce don, sachant qu'il restera anonyme par la suite.

        \subsection{Donataire}
            La situation du donataire est bien différente. En effet, celui-ci
            devrait déclarer les dons reçus, et verser des taxes sur ceux-ci.
            Or, comme nous l'avons vu précédemment, les dons de Gittip
            correspondent au mieux à des bénéfices non commerciaux.
            Vient ensuite le problème du régime fiscal. Celui ci dépendra
            de la situation de chaque donataire, mais le choix n'est pas
            simple, et rend encore plus ardu le processus de réception
            de dons.
            Enfin, subsiste le problème de la déclaration en elle-même.
            Effectivement le donateur étant anonyme, il sera difficile
            pour le donataire d'expliquer la provenance de cet argent.

    \section{Conclusion}
        Comme nous l'avons vu, la situation fiscale aux États-unis est moins
        contraignante que celle française  grâce aux
        facilités et abattements qu'offrent ceux-ci, en comparaison
        du relatif vide juridique qui existe en France sur la question.
        Le faible taux d'imposition qui existe aux États-Unis contraste
        fortement avec celui en vigueur en France, cependant avant qu'il
        existe un intérêt économique réel, il y a très peu de
        chances que le fisc s'intéresse réellement à ce type de revenus.
