\chapter*{Introduction}
\addcontentsline{toc}{chapter}{Introduction}

\paragraph{}
En France, la situation du logiciel libre semble excellente.
En effet, son adoption est en hausse, sa création est en croissance forte,
et notre pays en est le fleuron européen.
La récente introduction de Linux sur 37.000 postes de travail de la
gendarmerie nationale est un exemple frappant illustrant l'importance
du logiciel libre aussi bien dans le monde de l'entreprise que dans
le domaine publique.
Étudiants en informatique, nous sommes en outre particulièrement soucieux
de nous tenir informés de la situation de ce type de projets.
Dans un tel contexte, il n'est que naturel de se poser la question du
financement du logiciel open source.

\paragraph{}
Nous observons que le logiciel libre est, en grande majorité, créé,
développé et maintenu par des développeurs salariés, durant leur temps
libre, sans compensation financière. Il s'agit là d'un problème, d'une
part parce que cela constitue une barrière à l'entrée de nombreux talents
pouvant apporter une forte valeur en se concentrant sur leur domaine de
prédilection, mais aussi parce que la pérennité du logiciel libre est en
jeu : en l'absence de fonds, rien ne garanti que les logiciels utilisés
au sein même de nos institutions continueront à être maintenus.
Pour faire face à ce problème, plusieurs modèles, aux résultats plus ou
moins convainquant, ont été proposés; nous en étudieront un en
particulier, Gittip, et tâcherons de réponde à la question suivante.
Gittip a-t-il une chance de s'imposer comme un système de financement
reconnu et répandu en France aussi bien qu'à l'étranger ?

\paragraph{}
Pour ce faire, nous nous attaquerons tout d'abord à Gittip, en tant
qu'entreprise et que modèle économique, dans ses spécificités techniques.
Nous comparerons ensuite son modèle à d'autres modèles de financement
dignes d'intérêt et nous étudierons ensuite son rapport avec la législation
française et particulièrement à son volant fiscal pour déterminer si nos
lois constituent un frein ou un moteur à son adoption.
Nous terminerons enfin par une conclusion générale où nous synthétiserons
les résultats de nos recherches.
